\section{Введение}
(Тут речь идет о методиках и сложностях разработки, а надо подвести или лучше даже писать о мультитенантности)
В настоящее время все чаще внедряются сложные механизмы для облегчения различных процессов проектирования и производства в компаниях различного масштаба. Из-за достижения высоких технологий в различных областях промышленности очень сильно возросли требования к программному продукту. В связи с этим, реализация программ стала гораздо сложнее и требует больших усилий. 

Создавая программный продукт, работающий на удаленном компьютере - сервере и взаимодействующий с множеством клиентов, необходимо использовать множество средств, помогающих реализовывать, поддерживать и развивать разрабатываемый продукт. Из таких средств можно выделить различного рода программные библиотеки, программные и виртуальные среды, набор утилит для нужд программной реализации, ее отладки, тестов и запуска приложения.

Во все времена первостепенным средством для разработки являлась компьютерная операционная система ОС. Именно она содержала минимальный набор средств для создания продукта. Таким образом, имея только консоль и даже консольный текстовый редактор, можно было создавать программы. 
Но с развитием технологий требования усложнялись, подходы программирования становились сложнее и эффективнее. В наши дни недостаточно "пустой" операционной системы, чтобы реализовать сложное приложение. Программы наделили сложным графическим интерфейсом, они стали способны работать удаленно и использовать общие пользовательские ресурсы вместо личный локальных.
Изменились и подходы к созданию таких приложений. Перед началом разработки необходимо оценить важность работы, назвать и рассмотреть основные стадии ее реализации от начала развития до внедрения. Сегодня проблемы развития продукта решают различные методологии и практики разработки софта. 
Среди них известны такие, как 
\begin{itemize}
\item Водопадная методика
\item Спиралевидная методика
\item Инкрементальная методика
\item V-Model 
\item Agile
\item и другие
\end{itemize}
Для всех методик можно назвать общие этапы разработки ПО, и как минимум это тестирование и внедрение продукта. Если в процессе написания кода и сборки проекта используются известные среды разработки, то при выполнении вышеупомянутых этапов может не существовать подходящего решения для быстрой отладки, проверки работоспособности написанной программеы вскупе с другими компонентами системы. В данном случае необходимо реализовать собственные решения для эффективного тестирования.