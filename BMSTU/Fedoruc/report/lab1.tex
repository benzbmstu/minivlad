\section{Лабораторная работа 2 (вариант 57)}
\subsection{Задание}

Ферзь находится на поле D1 шахматной доски. Необходимо найти последовательность из пяти ходов, обеспечивающую прохождение ферзем максимального количества полей доски. При этом ферзь не имеет права останавливаться на любом поле более одного раза и пересекать уже пройденный маршрут последующими ходами.

\subsection{Иходный код}
\begin{verbatim}
     1  %?- target(S1,S2,S3,S4,S5).
     2  %create list of fields between X/Y and X1/Y1
     3  range(X/Y, X/Y, [X/Y]).
     4  range(X/Y, X1/Y1, [X/Y|T]):- X<X1, Y<Y1, X2 is X+1, Y2 is Y+1, range(X2/
Y2, X1/Y1, T).
     5  range(X/Y, X1/Y1, [X/Y|T]):- X<X1, Y>Y1, X2 is X+1, Y2 is Y-1, range(X2/
Y2, X1/Y1, T).
     6  range(X/Y, X1/Y1, [X/Y|T]):- X>X1, Y<Y1, X2 is X-1, Y2 is Y+1, range(X2/
Y2, X1/Y1, T).
     7  range(X/Y, X1/Y1, [X/Y|T]):- X>X1, Y>Y1, X2 is X-1, Y2 is Y-1, range(X2/
Y2, X1/Y1, T).
     8  range(X/Y, X1/Y1, [X/Y|T]):- X=X1, Y<Y1, X2 is X,   Y2 is Y+1, range(X2/
Y2, X1/Y1, T).
     9  range(X/Y, X1/Y1, [X/Y|T]):- X=X1, Y>Y1, X2 is X,   Y2 is Y-1, range(X2/
Y2, X1/Y1, T).
    10  range(X/Y, X1/Y1, [X/Y|T]):- X<X1, Y=Y1, X2 is X+1, Y2 is Y,   range(X2/
Y2, X1/Y1, T).
    11  range(X/Y, X1/Y1, [X/Y|T]):- X>X1, Y=Y1, X2 is X-1, Y2 is Y,   range(X2/
Y2, X1/Y1, T).
    12   
    13  %check whether X/Y is on board
    14  onboard(X/Y) :-
    15     0 < X, X < 9, 0 < Y, Y < 9.
    16   
    17  %integer is direction number (1-8), N is distance
    18  step(X/Y, X1/Y1, 1, N) :- X1 is X,     Y1 is Y + N, onboard(X/Y), onboar
d(X1/Y1).
    19  step(X/Y, X1/Y1, 2, N) :- X1 is X,     Y1 is Y - N, onboard(X/Y), onboar
d(X1/Y1).
    20  step(X/Y, X1/Y1, 3, N) :- X1 is X + N, Y1 is Y,     onboard(X/Y), onboar
d(X1/Y1).
    21  step(X/Y, X1/Y1, 4, N) :- X1 is X - N, Y1 is Y,     onboard(X/Y), onboar
d(X1/Y1).
    22  step(X/Y, X1/Y1, 5, N) :- X1 is X + N, Y1 is Y + N, onboard(X/Y), onboar
d(X1/Y1).
    23  step(X/Y, X1/Y1, 6, N) :- X1 is X + N, Y1 is Y - N, onboard(X/Y), onboar
d(X1/Y1).
    24  step(X/Y, X1/Y1, 7, N) :- X1 is X - N, Y1 is Y + N, onboard(X/Y), onboar
d(X1/Y1).
    25  step(X/Y, X1/Y1, 8, N) :- X1 is X - N, Y1 is Y - N, onboard(X/Y), onboar
d(X1/Y1).
    26   
    27  %8 directions are available
    28  direction(1).
    29  direction(2).
    30  direction(3).
    31  direction(4).
    32  direction(5).
    33  direction(6).
    34  direction(7).
    35  direction(8).
    36   
    37  %list all step variants
    38  variants([]).
    39  variants([H|T]) :-
    40     direction(H),
    41     variants(T).
    42   
    43  %N is distance
    44  %S is list of crossed fields
    45  queenstep(X/Y, X1/Y1, N, S):-
    46     variants([M,N]),
    47     step(X/Y, X1/Y1, M, N),
    48     range(X/Y, X1/Y1, S).
    49   
    50  uniq(Data,Uniques) :- sort(Data,Uniques).
    51   
    52  queenpath(X/Y, X1/Y1, X2/Y2, X3/Y3, X4/Y4, X5/Y5, L, S):-
    53     queenstep(X/Y,   X1/Y1, DL1, S1),
    54     queenstep(X1/Y1, X2/Y2, DL2, S2),
    55     queenstep(X2/Y2, X3/Y3, DL3, S3),
    56     queenstep(X3/Y3, X4/Y4, DL4, S4),
    57     queenstep(X4/Y4, X5/Y5, DL5, S5),
    58     L is DL1+DL2+DL3+DL4+DL5,
    59     append(S1, S2, T1),
    60     append(T1, S3, T2),
    61     append(T2, S4, T3),
    62     append(T3, S5, S6),
    63     uniq(S6, S),
    64     length(S, LTmp),
    65     LCalc is LTmp - 1,
    66     L = LCalc.
    67   
    68  target(X1/Y1, X2/Y2, X3/Y3, X4/Y4, X5/Y5):-
    69     queenpath(4/1, X1/Y1, X2/Y2, X3/Y3, X4/Y4, X5/Y5, 15, _).
    70


\end{verbatim}